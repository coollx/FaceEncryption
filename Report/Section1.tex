\section{Introduction}

Recent reports \cite{6227909} confirm the privacy risks associated with big data in public social media. The amount of user-generated content uploaded to the internet is increasing rapidly, but large corporations such as Google and Facebook have been misusing it without their knowledge \cite{10.1093/idpl/ipw026}. When consumers upload sensitive facial photographs, it is difficult to preserve their privacy. The company may use these pictures to develop their machine learning algorithms, which could result in privacy breaches. \textbf{How can we protect privacy when sharing facial images?}

In this research, we offer a viable privacy-protecting solution based on adversarial attacks and facial recognition technology. After detecting a face in a picture using a facial recognition model, we encrypt the face with noise generated by adversarial attacking model. The composed image may appear as clear as the original, but the facial recognition software will recognise a different individual in it.
