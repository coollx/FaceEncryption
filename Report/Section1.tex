\section{Introduction}

Recent research \cite{6227909} has verified that there are recognized privacy dangers related to the massive amounts of user-generated information published on public social media platforms. Internet behemoths like Google and Facebook have been secretly mining this data without the users' awareness or permission \cite{10.1093/idpl/ipw026}. When personal photos of people's faces are uploaded, this creates a severe privacy risk for those depicted. Companies may use these photographs to train machine learning algorithms, which could compromise users' personal information. \textbf{What measures can we take to ensure that individuals' privacy is maintained when exchanging photographs of their faces?}

We present a method that safeguards users' privacy when exchanging facial images by utilizing adversarial attacks and facial recognition algorithms. To accomplish this, we first use a facial recognition model to identify individuals in a picture, and then we encrypt their likenesses using noise produced by an adversarial attacking model. While the resulting image may have high visual similarities between human eyes, facial recognition software will fail to re-identify the right person in it. Such a facial encryption process can help people feel more comfortable sharing photographs online while maintaining their privacy.


